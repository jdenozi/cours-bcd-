\documentclass[12pt]{article}
\usepackage[english]{babel}
\usepackage{natbib}
\usepackage{url}
\usepackage[utf8x]{inputenc}
\usepackage{amsmath}
\usepackage{graphicx}
\graphicspath{{images/}}
\usepackage{parskip}
\usepackage{fancyhdr}
\usepackage{vmargin}
\usepackage{dirtytalk}
\setmarginsrb{3 cm}{2.5 cm}{3 cm}{2.5 cm}{1 cm}{1.5 cm}{1 cm}{1.5 cm}

\title{Economie et organisation }								% Title
\author{}								% Author
\date{}											% Date

\makeatletter
\let\thetitle\@title
\let\theauthor\@author
\let\thedate\@date
\makeatother

\pagestyle{fancy}
\fancyhf{}
\rhead{\theauthor}
\lhead{\thetitle}
\cfoot{\thepage}

\begin{document}

%%%%%%%%%%%%%%%%%%%%%%%%%%%%%%%%%%%%%%%%%%%%%%%%%%%%%%%%%%%%%%%%%%%%%%%%%%%%%%%%%%%%%%%%%

\begin{titlepage}
	\centering
    \vspace*{0.5 cm}
    \includegraphics[scale = 0.20]{logo.png}\\[1.0 cm]
    \textsc{\LARGE Faculté des Sciences de Montpellier\newline\newline Bioinformatique, Connaissance, Données}\\[2.0 cm]	% University Name
	\textsc{\Large HSMN115}\\[0.5 cm]				% Course Code
	\rule{\linewidth}{0.2 mm} \\[0.4 cm]
	{ \huge \bfseries \thetitle}\\
	\rule{\linewidth}{0.2 mm} \\[1.5 cm]
	

	
	
    
    
    
    
	
\end{titlepage}

%%%%%%%%%%%%%%%%%%%%%%%%%%%%%%%%%%%%%%%%%%%%%%%%%%%%%%%%%%%%%%%%%%%%%%%%%%%%%%%%%%%%%%%%%

\tableofcontents
\pagebreak

%%%%%%%%%%%%%%%%%%%%%%%%%%%%%%%%%%%%%%%%%%%%%%%%%%%%%%%%%%%%%%%%%%%%%%%%%%%%%%%%%%%%%%%%%
\section {Systèmes d'information}
\subsection{Système d'information}
\textbf{Système d'information:} 

\begin{minipage}{0.8\textwidth}
Ensemble organisé de ressources : matériel, logiciel, personnel, données, procédures permettant d'acquérir, traiter, stocker,
communiquer des informations dans des orgnaisations.
il existe une obligation légale de stocker les données dans un système aggréé, conforme aux normes du secret médical.
L'information est une représenatation de la réalité à laquelle les utilisateurs ont accès.
Les TI désignent l'acquisition, le traitement, le stockage et la transsmission de l'information. \\
\end{minipage}

Les bonnes procédures sont celles qui sont reconnues comme une ressource par les utilisateurs pour répondre au besoins d'un client. ex : on peut donner une
certaine autonomie à l'utilisateur pour gérer les problèmes qui surviennent.

Les logiciels de gestion de dossier médical en france sont conçus avec une conception restrictive du travail. 

La technologie choisie matérialise et fige le système d'information.

\subsection {Intéropérabilité/format de données: }
\textbf{Enjeux de l'intéropérabilité} :

\begin{minipage}{0.8\textwidth}
Capacité de plusieurs logiciels à échanger des données au format standard. Aucune norme n'est imposée au moment actuel dans le domaine de la santé. En
général, on utilise la norme ISO.
\end{minipage}
\begin{itemize}
\item vise à accorder les éléments techniques pour relier les systèmes informatiques entre eux
\item vise à préciser pour les inforlations échangées un sens compris par toutes les parties et à préserver ce sens.
\item vise à aligner les cadres législatifs et réglementaires pour conférer aux échanges de données le poids légal approuvé
\item vise à coordonner les processus 
\end{itemize}

En santé, le coût prohibe l'installation de systèmes redondants, l'absence de normes favorise cet état des lieux. Pour contraste : les systèmes informatiques sont gérés suivant des normes de sécurité, tout comme dans les banques.

\subsection{Esanté}
La e-santé ou santé numérique ou information numérique sur la santé recouvre les domaines de la santé qui font intervenir les technologies de l'information et de la communication (TIC). Le terme Santé est à prendre au sens large et comme l’OMS le souligne, ne concerne pas que les maladies et l’homme malade mais est aussi relatif à un état complet de bien-être physique, mental et social. Le développement de la e-santé s'appuie sur un domaine scientifique particulier, l'informatique médicale (ou informatique de santé), domaine qui a des liens étroits avec l’informatique mais dont les problématiques sont spécifiques du domaine santé. 
La création d'entreprise dans ce secteur a été favorisée au détriment des tickets d'entrée. Ainsi le niveau d'exigence est très faible.

\section{Notions pour comprendre les erreurs de qualités des données en santé}
\textbf {Le misfit technologies/organisation:} \\
\begin{minipage}{0.8\textwidth}
\textit{"Feeling that a system is interfering with the proper executions or orgaization operations"}. Il existe plusieurs categories de missfit. Il s'agit d'une inadequation entre technologie et organisation, il peut alors entrainer un misfit de donnees.\\
\end{minipage}

\textbf{Affordance:}

\begin{minipage}{0.8\textwidth}
La possiblité offerte à un être humain d'utiliser un élément de l'environnement en vue d'un objectif. On retrouve l'affordance fonctionnelle orientée vers l'objectif d'une action réalisée avec une TI. Elle peut être perçues ou non , selon la disposition dont on dispose.\\
\end{minipage}

\subsection{Qu'est ce qu'une organisation}
C'est un ensemble de moyen techniques et humains structurés et coordonénes pour oeuvrer vers un objectif commun.Ce sont les manières dont sont disposées les ressources et les coméptences individuelles et collectives d'une entreprise. On y retrouve une organisation publiques, privées, à but non lucratifniques et humains strututrés et coordonnées pour oeuvrer vers un objectif commun, il y a alors plusieurs parties prenantes, avec des intérêts divergents.

\subsection{Gouvernance des entreprises}
La gouvernance correspond à l'organisation et la répartition des pouvoirs entre les différences instances d'une entreprise et l'ensemble des procédures et structures mises en place pour :
\begin{itemize}
\item dirigier
\item gérer
\item contrôler les affaires d'une entreprise
\item l'autorité est liée à la détention de capital
\item régulation des reations entre  les parties prenantes
\end{itemize}

\textbf{Investisseur: }

\begin{minipage}{0.8\textwidth}
Un investisseur est une personne physique ou morale qui alloue une part de capital disponible dans l'attente d'un retour sur investissement. Il peut ainsi investir dans des actions, des obligations, des produits dérivés, des devises, des matières premières, de l'immobilier ou tout autre actif. Un investisseur peut investir sans distinction sur le marché primaire des titres nouvellement émis ou sur le marché secondaire (titres déjà émis), dans des titres de sociétés cotées ou non cotées.\\
\end{minipage}

\textbf{Capital}

\begin{minipage}{0.8\textwidth}
Le capital est une somme d'investissements utilisée pour en tirer un profit, c'est-à-dire un stock de biens ou de richesses nécessaires à une production. Cet emploi, courant en sciences économiques, en finance, en comptabilité, en sociologie et en philosophie, a néanmoins dans les domaines distincts des significations spécifiques. \\
\end{minipage}


\textbf{Le fond d'investissement}

\begin{minipage}{0.8\textwidth}
Un fonds de placement (ou fonds d'investissement) sont des organismes de détention collective d'actifs financiers.
\end{minipage}

\subsection{L'organisation selon Mintzberf(1982)}
\begin{itemize}
\item \textbf{Le sommet:} la direction générale ou du sommeil stratégique (y comprius le conseil d'administration).\\
\item \textbf{La ligne hiérarchique : }l'encadrement et la supervision opérationnelle.\\
\item \textbf{Le centre opérationnel:} l'endroit ou les collaborateurs accomplissent le travail.\\
\item \textbf{La technostructure:} des analyses qui standardisent le travail des aures: Analystes du travail, analyste planification.\\
\item \textbf{La logistique ou le support :} La logistique est l'activité qui a pour objet de gérer les flux physiques, et les données (informatives, douanières et financières) s'y rapportant, dans le but de mettre à disposition les ressources correspondant à des besoins (plus ou moins) déterminés en respectant les conditions économiques et légales prévues, le degré de qualité de service attendu, les conditions de sécurité et de sûreté réputées satisfaisantes.
\end{itemize}

\subsection{Types d'organisation: }
\textbf{Structure simple:}
\begin{itemize}
\item Supervision directe par le chef
\item Manageemnt autoritaire ou paternaliste
\item Autorité sur la tradition
\end{itemize}

\textbf{Organisation bureaucratique}
\begin{itemize}
\item Management directif et centralisé
\item Coordination par la standardisation des règles et ou des produits.
\item Autorités basée sur le statut
\item Ligne hiérarchique
\end{itemize}

\textbf{Organisation divisionalisée}
\begin{itemize}
\item Un siège centralisateur
\item Des divisions clones
\item Conflits fréquents entre managers des divisions et managers stratégiques
\end{itemize}


\textbf{Organisation adhocratique}
\begin{itemize}
\item Management démocratique
\item Pas de hiérarchie informel
\end{itemize}


\hspace{1 cm}--- Ju

\newpage
\bibliographystyle{plain}
\bibliography{biblist}
enter
\end{document}